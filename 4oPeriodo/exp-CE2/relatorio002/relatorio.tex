\documentclass[a4paper,12pt,oneside,openany,table,xcdraw]{article}

\usepackage{setspace}
\usepackage{multirow}
\usepackage{hyperref}
\usepackage{caption}
\usepackage{indentfirst}

\usepackage[brazilian]{babel}
\usepackage[utf8x]{inputenc}
\usepackage{amsmath, graphicx, mathptmx, enumerate}
\usepackage{float, verbatim}
\usepackage[colorinlistoftodos]{todonotes}
\usepackage{makeidx} % Para o sumário
\usepackage{geometry}

\geometry{a4paper, hmargin={3cm, 3cm}, vmargin={3cm, 2cm} }
\setlength{\parindent}{1.0cm}

\begin{document}
\newcommand{\thedepartment}{Faculdade de Engenharia Elétrica}
\newcommand{\thecourse}{FEELT}
\newcommand{\thetitle}{TENSÕES, CORRENTE E POTÊNCIAS EM CIRCUITO SÉRIE, FATOR DE POTÊNCIA E CORRENTE ALTERNADA SENOIDAL - USO DE MEDIDORES ANALÓGICOS E DIGITAIS - VATÍMETRO ANALÓGICO}
\newcommand{\thetype}{Relatório da Disciplina de Circuitos Elétricos II}
\newcommand{\theproftitle}{Bacharel em Engenharia Elétrica}
\newcommand{\thestudent}{Lesly Viviane Montúfar Berrios\\
\centering11811ETE001}
\newcommand{\theadvisor}{Prof. Wellington Maycon Santos Bernardes}
\newcommand{\thecity}{Uberlândia}

\thispagestyle{empty}\newcommand*{\themonth}{\ifthenelse{\the\month < 2}{Janeiro }
                  {\ifthenelse{\the\month < 3}{Fevereiro }
                  {\ifthenelse{\the\month < 4}{Março }
                  {\ifthenelse{\the\month < 5}{Abril }
                  {\ifthenelse{\the\month < 6}{Maio }
                  {\ifthenelse{\the\month < 7}{Junho }
                  {\ifthenelse{\the\month < 8}{Julho }
                  {\ifthenelse{\the\month < 9}{Agosto }
                  {\ifthenelse{\the\month < 10}{Setembro }
                  {\ifthenelse{\the\month < 11}{Outubro }
                  {\ifthenelse{\the\month < 12}{Novembro }{Dezembro }}}}}}}}}}}}
                  
\begin{titlepage}
\begin{center}

	\vspace{-0.5cm}

  \begin{figure}[hbt!]
		\begin{center}
		   \includegraphics[width=2.8cm]{ufu-logo.png}
		\end{center}
	\end{figure}
 	%\vspace{-4cm}

%\begin{doublespacing}

  \Large{\textbf{Universidade Federal de Uberlândia}}\\
  \large{\thedepartment}\\
  \large{\thecourse}\\


\vspace{6cm}
  \par
  \large\textbf{\thetitle}
\vspace{6cm} 

%\end{doublespacing}
  \par
  \thetype\\
  por\\
  %\hspace{2cm}\large{}\\

\vspace{1cm}
\par\vfil
  \normalsize{\thestudent}\\ [2cm]
  \theadvisor

\par\vfill
  \thecity, \themonth / \the\year

\end{center}

\end{titlepage}

%% Comeca o documento !

\onehalfspacing
\tableofcontents % sumário
\newpage

\section{Objetivos} % 2,5%
Montar um circuito série \emph{RLC}, energizá-lo com tensão alternada senoidal, realizar medições usando
equipamentos analógicos e digitais, efetuar desenvolvimentos teóricos e cálculos numéricos confrontando os
resultados teóricos com aqueles obtidos experimentalmente. Comparar a potência ativa
obtida pelo vatímetro analógico com o valor obtido no medidor digital.

\section{Introdução teórica} % 5%

\subsection{Análise do circuito}
O circuito a ser analisado neste experimento é descrito na Figura \ref{circuito} e do conhecimento teórico de circuitos em série tem-se os cálculos descritos pelas Equações (\ref{Z}) e (\ref{I}).

\begin{figure}[H]
\centering
\captionsetup{font=scriptsize}
\includegraphics[width=11cm]{circuito}
\caption{Montagem experimental.}
\label{circuito}
\end{figure}

A impedância total na forma fasorial é descrita como na Equação \ref{Z}, assim tomando-se o módulo é possível descrever a corrente como na Equação \ref{I}.
\begin{equation}\label{Z}
\dot{Z}=(R+R_{B}) + j\, (X_{L_{B}}+ X_{C})
\end{equation}
\begin{gather*}
Z = \sqrt{(R+R_{B})^2 + (X_{L_{B}}+ X_{C})^2 } \\
V = Z\, I
\end{gather*}
\begin{equation}\label{I}
I = \dfrac{V}{\sqrt{(R+R_{B})^2 + (X_{L_{B}}+ X_{C})^2 }}
\end{equation}

\subsection{Potências Eficazes}
As potências ativa, reativa e aparente eficazes podem ser calculadas, respectivamente, pelas Equações (3), (4) e (5). 
\begin{gather}
P=V_{ef}\cdot I_{ef}\cdot cos\theta\\
Q=V_{ef}\cdot I_{ef}\cdot sen\theta\\
S=V_{ef}\cdot I_{ef}
\end{gather}

\section{Preparação}
\subsection{Materiais e ferramentas} % 2,5%
\begin{enumerate}[1 - ]
\item \emph{Fonte}\\
Alimentará todo o circuito.

\item \emph{Variador de tensão (Varivolt)}\\
O equipamento permitirá obter o valor desejado de corrente a partir da regulagem correta da tensão fornecida pela fonte. Também chamado de autotransformador.

\item \emph{Medidor eletrônico KRON Mult K}\\
Possibilita encontrar a medição da potência
real (P) - vatímetro, reativa (Q) e aparente (S) do circuito. Ele também possui função de cofasímetro, instrumento elétrico que mede o fator de potência (fp, $cos\theta$) ou o ângulo da impedância $\theta$ do circuito, para um circuito com a impedância $Z = Z\angle \theta$.

\item \emph{Conectores}\\
Foram utilizadas pontas de provas para a verificação das grandezas nos multímetros e pontas de prova específicas para multímetro. Para as conexões no circuito foi utilizado majoritariamente cabos banana-banana.

\item \emph{Multímetro}\\
Utilizado para medir a resistência R, capacitância C e gradezas do conjunto L e $R_L$ especificados no experimento.

\item \emph{Amperímetro analógico AC}\\
Instrumento de maior precisão.

\item \emph{Voltímetro analógico AC}\\
Instrumento de maior precisão.

\item \emph{Osciloscópio}\\
Utilizado obter informações da forma de onda ($V_{pp}$,$V_{max}$, $V_{rms}$).

\item \emph{Reostato R}\\
Reostato com potência nominal de aproximadamente 1kW.

\item \emph{Capacitor C}\\
Reostato com potência nominal de aproximadamente 1kW.

\item \emph{Bobina B}\\
O valor medido da indutância da bobina B (reator para lâmpada vapor de sódio) realizada
recentemente (Agosto/2019) é de 160 mH e resistência interna de 3,8 ohms.

\end{enumerate}

\subsection{Montagem} % 2,5%

%\begin{enumerate}[1)]
%\item \emph{Montando o circuito}\\
Realize a montagem informada na Figura \ref{fig1}, com os parâmetros R, C, L, RL, V e f (preenchendo as
Tabelas 1 e 2). 

\begin{figure}[H]
\centering
\captionsetup{font=scriptsize}
\includegraphics[width=14.5cm]{fig1}
\caption{Montagem experimental.}
\label{fig1}
\end{figure}

%\end{enumerate}

\section{Análise sobre segurança} % 2,5%
Os óculos de segurança são Equipamentos de Proteção Individual (EPIs) e são utilizados para a proteção da área ao redor dos olhos contra qualquer tipo de detrito estranho, que possa causar irritação ou ferimentos. Também protegem contra faíscas, respingos de produtos químicos, detritos, poeira, radiação e etc \cite{safe}.
É importante a utilização desse equipamento durante os experimentos a fim de evitar qualquer dano, além de preparar o profissional para o manejo correto e seguro de qualquer equipamento.
Além disso, foi de extrema importância a presença do professor ou técnico na verificação da montagem do circuito antes de energizá-lo. Assim, reduziu-se riscos de curtos-circuitos ou sobrecarga na rede.

\section{Cálculos, análise dos resultados e questões} % (quando houver) (70%)

\subsection{Comparação das medidas experimentais}
\begin{enumerate}[1 - ]
\item Complete a Tabela \ref{tab1} com os dados do Caso A, sendo $V_{ef}=100V$ e $R=100\Omega$ (teórico).

\begin{table}[h]
\centering
\def\arraystretch{1.35}
\captionsetup{font=scriptsize}
\captionof{table}{Parâmetros reais da montagem do primeiro caso.} \label{tab1}
\begin{tabular}{|c|c|c|c|c|c|}
\hline
$R [\Omega]$ & $C [\mu F]$ & $L [mH]$ & $R_L [\Omega]$ & $V [volts]$ & $f [Hz]$ \\ \hline
       100      &    45,9     &    160      &         3,8       &      99,4       &     59,95     \\ \hline
\end{tabular}
\end{table}

\item Complete a Tabela \ref{tab2} com os dados do Caso B, sendo $V_{ef}=50V$ e $R=20\Omega$ (teórico). \\

\begin{table}[h]
\centering
\def\arraystretch{1.35}
\captionsetup{font=scriptsize}
\captionof{table}{Parâmetros reais da montagem do segundo caso.} \label{tab2}
\begin{tabular}{|c|c|c|c|c|c|}
\hline
$R [\Omega]$ & $C [\mu F]$ & $L [mH]$ & $R_L [\Omega]$ & $V [volts]$ & $f [Hz]$ \\ \hline
       20      &    45,9     &    160      &         3,8       &       49,39      &     60,00     \\ \hline
\end{tabular}
\end{table}

\item Ajuste a tensão de saída do autotransformador (varivolt) de maneira a obter a tensão solicitada para o voltímetro e anote os valores medidos na Tabela \ref{tab3} (para ambos os casos, A e B). Os resultados são obtidos por meio dos cálculos apresentados na introdução teórica. \\
\begin{table}[H]
\centering
\def\arraystretch{1.35}
\captionsetup{font=scriptsize}
\captionof{table}{Erro percentual das duas montagens.} \label{tab3}

\resizebox{\textwidth}{!}{ %
\begin{tabular}{|c|c|c|c|c|c|c|c|c|c|c|c|c|}
\hline
\multirow{3}{*}{Valores} & \multicolumn{9}{c|}{Medições}                                                                         & \multicolumn{3}{c|}{Cálculos}          \\ \cline{2-13} 
                         & $V_{ef}$ & I       & $cos\theta$ & $V_R$   & $V_C$   & $V_{(L+R_L)}$ & P       & S        & Q         & $\theta^{[1]}$ & $S^{[2]}$ & $Q^{[3]}$ \\ \cline{2-13} 
                         & {[}V{]}  & {[}A{]} & {[}fp{]}    & {[}V{]} & {[}V{]} & {[}V{]}       & {[}W{]} & {[}VA{]} & {[}VAr{]} & $[^{\circ}]$   & {[}VA{]}  & {[}Var{]} \\ \hline
\multicolumn{13}{|c|}{Caso A}                                                                                                                                             \\ \hline
Medidos                  & 99,40    & 0,932   & 0,988       & 93,10   & 54,36   & 69,40         & 90,90   & 92,01    & 14,23  &   8,89           & 92,64     & 14,25     \\ \hline
Calculados               & 100      & 0,963   & 0,999       &   96,30      &  55,65    & 58,19  & 96,20    & 96,30    & 4,30   &   2,56           &  96,30    &  4,30         \\ \hline
Erros (\%)               & -0,60       & -3,34   & -1,113   &  -3,44  &  -2,37   &  16,15        & -5,83     & -3,95    & 69,77 &   71,20          & -3,95     &  69,82         \\ \hline
\multicolumn{13}{|c|}{Caso B}                                                                                                                                             \\ \hline
Medidos                  & 49,39    & 1,702   & 0,873      & 34,16   & 99,8    & 124,00        & 73,10   & 84,00    & 41,39     & 29,19          & 84,06     & 41,38     \\ \hline
Calculados               & 50,00   & 2,089    & 0,994     & 41,78   &120,72 & 126,24        & 103,82  & 104,45   &  11,41         &      6,27      &    104,45     &  11,45        \\ \hline
Erros (\%)               &-1,24     &-22,74   &-13,91     & -22,30  & -20,97 & -1,81          & -42,02  & -24,34    &  72,43         &   78,48       &    -24,26       &  72,32       \\ \hline
\end{tabular} % 
}
\end{table}

\noindent\text{[1]} Calcule o valor medido de $\theta$ à partir do fator de potência, ou seja, $\theta = arccos(fp)$. \\
\noindent\text{[2]} Calcule a potência aparente S à partir dos valores medidos para V e I, ou seja, $S=V\times I$. \\
\noindent\text{[3]} Calcule a potência reativa Q à partir do triângulo de potência, ou seja, $Q^2=S^2-P^2$. 

Calculando-se as impedâncias sobre o capacitor e indutor tem-se, respectivamente, $X_{C}=57,79\Omega$ e $X_{L_B}=60,31\Omega$.  Ademais, para a bobina $Z_B=\sqrt{R_B^2+L_B^2}=60,43\Omega$. O cálculo do fp foi relizado pela relação do triângulo de impedâncias.

Para o caso A, do cálculo do módulo da impedância por meio da Equação (\ref{Z}) tem-se $Z=103,8284\Omega$, logo $I=0,963A$.
Já para o caso B, do cálculo do módulo da impedância por meio da Equação (\ref{Z}) tem-se $Z=23,9339\Omega$, logo $I=2,089A$.

\item Ligue o osciloscópio (canal CH1), automatize o trigger e colete $V_{pp}$, $V_m$ e $V_{rms}$. Registre a imagem.
Use a função MEASURE $>$ TODAS MED para o equipamento realizar os cálculos práticos.

\begin{figure}[H]
\centering
\captionsetup{font=scriptsize}
\includegraphics[width=11cm]{osc1}
\caption{Imagem do osciloscópio para o Caso A.}
\label{osc1}
\end{figure}
\begin{figure}[H]
\centering
\captionsetup{font=scriptsize}
\includegraphics[width=11cm]{osc2}
\caption{Imagem do osciloscópio para o Caso B.}
\label{osc2}
\end{figure}

\item Comparação dos equipamentos de medição analógicos e digitais \\
\begin{table}[H]
\centering
\def\arraystretch{1.35}
\captionsetup{font=scriptsize}
\captionof{table}{Comparativo das medições para o Caso A.} \label{cA}
\begin{tabular}{c|c|c|}
\cline{2-3}
                                              & \textit{V {[}V{]}} & \textit{I {[}A{]}} \\ \hline
\multicolumn{1}{|c|}{KRON Mult K}             & 99,40              & 0,932              \\ \hline
\multicolumn{1}{|c|}{Analógico}               & 102,00             & 0,930              \\ \hline
\multicolumn{1}{|c|}{Osciloscópio}            & 106,00             & (1,021)             \\ \hline % V = Z * I (tudo rms)
\multicolumn{1}{|c|}{Erros Analógico (\%)}    & -2,62              & 0,21               \\ \hline
\multicolumn{1}{|c|}{Erros Osciloscópio (\%)} & -6,64              & -9,540             \\ \hline
\end{tabular}
\end{table}

\begin{table}[H]
\centering
\def\arraystretch{1.35}
\captionsetup{font=scriptsize}
\captionof{table}{Comparativo das medições para o Caso B.} \label{cB}
\begin{tabular}{c|c|c|}
\cline{2-3}
                                              & \textit{V {[}V{]}} & \textit{I {[}A{]}} \\ \hline
\multicolumn{1}{|c|}{KRON Mult K}             & 49,39             & 1,732           \\ \hline
\multicolumn{1}{|c|}{Analógico}               & 50,08               & 1,65             \\ \hline
\multicolumn{1}{|c|}{Osciloscópio}            & 88,00              & (3,677)            \\ \hline
\multicolumn{1}{|c|}{Erros Analógico (\%)}   & 1,39            & 4,73              \\ \hline
\multicolumn{1}{|c|}{Erros Osciloscópio (\%)} & -78,17        & -112,3                   \\ \hline % Nao sei o que aconteceu (-78,17)
\end{tabular}
\end{table}

\vspace{0.3cm}

\item Manejo do equipamento \emph{KRON Mult K}\\
A regulagem do equipamento utilizando-se os parâmetros TP (Transformador de Potencial), TC (Transformador de Corrente) e TL (Transformador de Ligação) foi essencial, uma vez que é preciso informar ao equipamento que se trata de um circuito monofásico (1 fase + neutro), para isso configura-se $TL=0002$. TP e TC foram regulados a partir dos equipamentos analógicos pela relação descrita pela Equação \ref{TP} e \ref{TC}e conseguiu-se $TP=1,00$ e $TC=1,01$, para os quais os valores no equipamento digital também correspondem ao do analógico.
\begin{equation}\label{TP}
\dfrac{{TP}_{antigo}}{{TP}_{novo}}=\dfrac{V_{KRON}}{V_{analogico}}
\end{equation}
\begin{equation}\label{TC}
\dfrac{{TC}_{antigo}}{{TC}_{novo}}=\dfrac{I_{KRON}}{I_{analogico}}
\end{equation}

\end{enumerate}

\subsection{Questões}
\begin{enumerate}[1)]
\item A potência ativa lida no mediador KRON Mult K apresenta informação incorreta em relação ao vatímetro
analógico. Aponte as possíveis causas.\\
A possível causa pode ser devido ao
equipamento estar com defeito, mal contato nos cabos, TC e TP desajustados ou TL desconfigurado.

\item Por que dependendo do tipo da ligação do vatímetro, seu ponteiro indicador deflete em sentido “negativo”?\\
Uma vez que ligado de forma incorreta ou invertida pode detectar uma potência negativa e essa polarização negativa é representada pela deflexão do
ponteiro no sentido inverso.

\item Quais as vantagens da utilização do mediador KRON Mult K frente aos medidores analógicos? Discuta a
respeito de espaço físico empregado para a utilização dos equipamentos bem como o tempo de montagem.
Pesquise também sobre custos para aquisição.\\
Medidores digitais como o KRON são mais precisos e facilita a leitura para o usuário. Ademais permite a medição de várias grandezas em um só equipamento, o que economiza tempo. Já os equipamentos analógico tendem
a ocupar mais espaço, além de demandarem a utilização de muitos cabos em suas conexões. Com relação a custos, os aparelhos analógicos são consideravelmente
mais baratos que o KRON Mult K.

\item Considerando que a escala percentual do reostato esteja correta, qual é o efeito físico no amperímetro,
multímetro e vatímetro se o usuário excursiona de 25\% para 50\% da resistência nominal?\\

Ao reduzir a resistência pela metade, a potência $P=V\cdot I$ também cairá pela metade, assim como a leitura no amperímetro a medida será reduzida pela metade,
no voltímetro não haverá alterações e no wattímetro teremos também a metade, devido à ligação em paralelo.

\item Explique a importância do transformador de potencial e de corrente no medidor KRON Mult K.\\
O TP e TC são de maior importância em circuitos trifásicos com transformadores, uma vez que neste experimento, como é um circuito monofásico, a relação deverá estar próxima de 1, dado que só há influência de erros do próprio equipamento KRON.

\item Qual é a importância de AAUX e VAUX? Neste roteiro, é necessária a permanência constante desses medidores
ou podem ser eliminados sem prejuízo? Se sim, em qual momento?\\
São usados para calibrar os valores de TL, TC e TP do KRON, mas podem ser
retirados a qualquer momento.

\item Nota-se que muitos medidores analógicos possuem um espelho logo abaixo da escala graduada. Explique o
motivo.\\
O espelho existe para que o usuário consiga fazer uma leitura mais precisa. O
valor correto é aquele em que o ponteiro e a imagem refletida no espelho coincidam.
\end{enumerate}


\section{Simulação computacional} % (10%);

\subsection{Caso A}
Da simulação computacional tem-se as Figuras \ref{corrente1}, \ref{tensao-R1}, \ref{tensao-C1} e \ref{tensao-B1}.
\begin{figure}[H]
\centering
\captionsetup{font=scriptsize}
\includegraphics[width=11cm]{corrente1}
\caption{Corrente do circuito.}
\label{corrente1}
\end{figure}
\begin{figure}[H]
\centering
\captionsetup{font=scriptsize}
\includegraphics[width=11cm]{tensao-R1}
\caption{Tensão no resistor R.}
\label{tensao-R1}
\end{figure}
\begin{figure}[H]
\centering
\captionsetup{font=scriptsize}
\includegraphics[width=11cm]{tensao-C1}
\caption{Tensão no capacitor C.}
\label{tensao-C1}
\end{figure}
\begin{figure}[H]
\centering
\captionsetup{font=scriptsize}
\includegraphics[width=11cm]{tensao-B1}
\caption{Tensão na bobina B.}
\label{tensao-B1}
\end{figure}

\subsection{Caso B}
Da simulação computacional tem-se as Figuras \ref{corrente2}, \ref{tensao-R2}, \ref{tensao-C2} e \ref{tensao-B2}.
\begin{figure}[H]
\centering
\captionsetup{font=scriptsize}
\includegraphics[width=11cm]{corrente2}
\caption{Corrente do circuito.}
\label{corrente2}
\end{figure}
\begin{figure}[H]
\centering
\captionsetup{font=scriptsize}
\includegraphics[width=11cm]{tensao-R2}
\caption{Tensão no resistor R.}
\label{tensao-R2}
\end{figure}
\begin{figure}[H]
\centering
\captionsetup{font=scriptsize}
\includegraphics[width=11cm]{tensao-C2}
\caption{Tensão no capacitor C.}
\label{tensao-C2}
\end{figure}
\begin{figure}[H]
\centering
\captionsetup{font=scriptsize}
\includegraphics[width=11cm]{tensao-B2}
\caption{Tensão na bobina B.}
\label{tensao-B2}
\end{figure}

\section{Conclusões} % (no mínimo 10 linhas) (5%);
Este experimento trata-se da análise de um circuito série \emph{RLC} energizado com tensão senoidal, descrito pela Figura \ref{fig1}. Calculou-se as impedâncias, corrente, tensão sobre cada componente a partir da análise das malhas, além das potências eficazes, por meio da análise teórica. Assim, foi possível a comparação com os valores obtidos experimentalmente tanto com medidores analógicos quanto com os digitais, o que é descrito pelas Tabelas \ref{cA} e \ref{cB}. Também foi importante a configuração do equipamento \emph{KRON Mult K} por meio dos parâmetros TL, TP e TC. 

Finalmente, a simulação computacional permitiu obter as medições com os dados experimentais dos componentes, dessa forma, identificou-se os erros associados aos equipamentos de medidas, seja por desregulagem ou erro do olho humano. Agrega-se a importância dos Equipamentos de Segurança Individual (EPI), uma vez que a utilização dos óculos durante o manejo dos componentes, mesmo que aparentemente ainda não sejam potenciais situações de risco, permite a criação do hábito de proteção e prevenção, uma característica essencial para o engenheiro na área de trabalho.


\newpage
\begin{thebibliography}{9} 
% Introdução
\bibitem{irwin}
    J. D. Irwin,
    “Análise de Circuitos Em Engenharia”, Pearson, $4^a$ Ed., 2000.

\bibitem{boylestad}
    R. L. Boylestad,
    “Introdução À Análise de Circuitos”, Pearson, $10^a$ Ed., 2004.

\bibitem{safe}
    SafetyTrabi,
    “Óculos de segurança: Saiba quando utilizar este EPI”, SafetyTrab, 2019.
 Disponível em:
 \url{https://www.safetytrab.com.br/blog/oculos-de-seguranca/}. Acesso em: ago. 2019.


\end{thebibliography}
\end{document}
