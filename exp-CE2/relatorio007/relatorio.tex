\documentclass[a4paper,12pt,oneside,openany,table,xcdraw]{article}

\usepackage{setspace}
\usepackage{multirow}
\usepackage{hyperref}
\usepackage{caption}
\usepackage{indentfirst}
\usepackage{tikz} %% fasores
\usetikzlibrary{arrows,arrows.meta,quotes,angles}
\usepackage{siunitx}

\usepackage[brazilian]{babel}
\usepackage[utf8x]{inputenc}
\usepackage{amsmath, graphicx, subfig, enumerate}
\usepackage{float, verbatim}
\usepackage[colorinlistoftodos]{todonotes}
\usepackage{makeidx}
\usepackage{geometry}

\graphicspath{{img/}}
\geometry{a4paper, hmargin={3cm, 3cm}, vmargin={3cm, 2cm} }
\setlength{\parindent}{1.0cm}
\captionsetup{font=small}

\begin{document}
\newcommand{\thedepartment}{Faculdade de Engenharia Elétrica}
\newcommand{\thecourse}{FEELT}
\newcommand{\thetitle}{VERIFICAÇÃO DA SEQUÊNCIA DE FASES DAS TENSÕES}
\newcommand{\thetype}{Relatório da Disciplina de Experimental de Circuitos Elétricos II}
\newcommand{\theproftitle}{Bacharel em Engenharia Elétrica}
\newcommand{\thestudent}{Lesly Viviane Montúfar Berrios\\
\centering11811ETE001}
\newcommand{\theadvisor}{Prof. Wellington Maycon Santos Bernardes}
\newcommand{\thecity}{Uberlândia}

\thispagestyle{empty}\newcommand*{\themonth}{\ifthenelse{\the\month < 2}{Janeiro }
                  {\ifthenelse{\the\month < 3}{Fevereiro }
                  {\ifthenelse{\the\month < 4}{Março }
                  {\ifthenelse{\the\month < 5}{Abril }
                  {\ifthenelse{\the\month < 6}{Maio }
                  {\ifthenelse{\the\month < 7}{Junho }
                  {\ifthenelse{\the\month < 8}{Julho }
                  {\ifthenelse{\the\month < 9}{Agosto }
                  {\ifthenelse{\the\month < 10}{Setembro }
                  {\ifthenelse{\the\month < 11}{Outubro }
                  {\ifthenelse{\the\month < 12}{Novembro }{Dezembro }}}}}}}}}}}}
                  
\begin{titlepage}
\begin{center}

	\vspace{-0.5cm}

  \begin{figure}[hbt!]
		\begin{center}
		   \includegraphics[width=2.8cm]{ufu-logo.png}
		\end{center}
	\end{figure}
 	%\vspace{-4cm}

%\begin{doublespacing}

  \Large{\textbf{Universidade Federal de Uberlândia}}\\
  \large{\thedepartment}\\
  \large{\thecourse}\\


\vspace{6cm}
  \par
  \large\textbf{\thetitle}
\vspace{6cm} 

%\end{doublespacing}
  \par
  \thetype\\
  por\\
  %\hspace{2cm}\large{}\\

\vspace{1cm}
\par\vfil
  \normalsize{\thestudent}\\ [2cm]
  \theadvisor

\par\vfill
  \thecity, \themonth / \the\year

\end{center}

\end{titlepage}

%% Comeca o documento !

\onehalfspacing
\tableofcontents % sumário
\newpage

\section{Objetivos} % 2,5%
Pretende-se verificar experimentalmente conceitos teóricos de como encontrar a correta sequência de
fase diante da ausência de um sequencímetro (método do voltímetro).

\section{Introdução teórica} % 5%

O sequencímetro é um instrumento de medida elétrica analógica ou digital que tem por  finalidade a verificação da sequência de fases de um motor trifásico (circuito alimentado por corrente alternada), ou seja, indica a fase aberta e o sentido de rotação do motor. Na Figura \ref{intro:fig1}, é observado um fasímetro, que possui a mesma função, havendo poucas diferenças, entre elas, estão a tensão de entrada e a faixa de frequência. Sobre seu funcionamento diz-se que, a partir do momento em que o sequencímetro detecta a passagem por zero (pulso positivo de curta duração) de cada fase é aplicado em um circuito sequencial feito com flip-flop e indica a sequencia da rede \cite{UNIR}.

Circuitos desequilibrados podem ser utilizados para a verificação de sequência de fases em certo sistema elétrico, na ausência de um sequencímetro. Para isso

\vspace{0.3cm}
\begin{figure}[H]
\centering
\includegraphics[width=13cm]{fasimetro}
\caption{Fasímetro com indicador led 690 volts - MFA-862 \cite{fig1}.}
\label{intro:fig1}
\end{figure}
\vspace{0.3cm}


\section{Preparação}
\subsection{Materiais e ferramentas} % 2,5%
\begin{enumerate}[1 -]
\item \emph{\textbf{Fonte:}}
Alimentará todo o circuito. Possui frequência de $60\ Hz$.

\item \emph{\textbf{Regulador de tensão (Varivolt):}}
Também chamado de autotransformador, permitirá obter o valor desejado de corrente a partir da regulagem correta da tensão fornecida pela fonte.

\item \emph{\textbf{Conectores:}}
Para as conexões no circuito foi utilizado majoritariamente cabos banana-banana.

\item \emph{\textbf{Medidor eletrônico KRON Mult K:}}
Possibilita encontrar a medição da potência real (P) - vatímetro, reativa (Q) e aparente (S) do circuito. Ele também possui função de cofasímetro, instrumento elétrico que mede o fator de potência (fp, $cos\theta$) ou o ângulo da impedância $\theta$ do circuito, para um circuito com a impedância $Z = Z\angle \theta$.

\item \emph{\textbf{Amperímetro analógico AC:}}
Instrumento utilizado para acompanhar visualmente o aumento da corrente.

\item \emph{\textbf{Resistor de $50\Omega$:}}
Carga resistiva para compor a carga do circuito trifásico.

\item \emph{\textbf{Capacitor de $45,9\mu F$:}}
Sendo sua resistência quase nula, portanto desprezível nessa aplicação (Esquenta pouco, logo dissipa menos energia).
\end{enumerate}

\vspace{0.2cm}
\subsection{Montagem} % 2,5%

\subsubsection{Carga em estrela com neutro conectado} \label{m1:montagem}
A montagem utilizada observa-se na Figura \ref{m1:esquema}, a qual ilustra o circuito na sequência de fases ABC. Pretende-se com este circuito investigar-se acerca do efeito do neutro em circuitos trifásicos desequilibrados. Usou-se TL=0000, TC=TP=1 como configurações no medidor \emph{Kron}. Aplica-se uma tensão linha $V_L=100V$ com o auxílio do \emph{Varivolt}, em frequência de 60Hz. Ademais, a carga desequilibrada possui os seguintes parâmetro: $R=50\Omega$, $R_L=3,8\Omega$, $L=166mH$ e $C=45,9 \mu F$.

\vspace{0.2cm}
\begin{figure}[H]
\centering
\includegraphics[width=13cm]{m1-circuito}
\caption{Circuito esquemático da montagem 1.}
\label{m1:esquema}
\end{figure}

\subsubsection{Carga em estrela com neutro isolado} \label{m2:montagem}
Com os mesmos parâmetros de impedância e tensão de entrada, porém agora com neutro isolado, mantém-se a configuração do medidor \emph{Kron}. Entretanto, nessa situação espera-se deslocamento da tensão no neutro, ou seja, $V_{n'n} \ne 0$.

\vspace{0.1cm}
\begin{figure}[H]
\centering
\includegraphics[width=13cm]{m2-circuito}
\caption{Circuito esquemático da montagem 2.}
\label{m2:esquema}
\end{figure}
\vspace{0.1cm}

\subsubsection{Carga em triângulo desequilibrado} \label{m3:montagem}
Agora, na conexão em triângulo e sem neutro, a configuração TL é diferente (TL=0048, $3\phi$ sem Neutro). Nessa montagem, tem-se tensão de entrada $V_{AB}=50V$, a fim de evitar-se correntes próximas ou superiores a $1,8A$ no medidor \emph{Kron}. 

\vspace{0.15cm}
\begin{figure}[H]
\centering
\includegraphics[width=13cm]{m3-circuito}
\caption{Circuito esquemático da montagem 3.}
\label{m3:esquema}
\end{figure}
\vspace{0.2cm}

\section{Dados Experimentais}

\subsection{Carga em estrela com neutro conectado} \label{m1:dados}
Dos dados da Tabela \ref{m1:dados:abc}, ainda tem-se $P_T = 115,5$W, $Q_T = 61,42$VAr e $S_T = 144,24$VA. Enquanto que para sequência de fases CBA (Tabela \ref{m1:dados:cba}), $P_T = 118,16$W, $Q_T = 60,7$VAr e $S_T = 155,79$VA.

\begin{table}[H]
\centering \small \def\arraystretch{1.32}
\caption{Dados experimentais da primeira montagem em sequência ABC.}
\label{m1:dados:abc}
\resizebox{\textwidth}{!}{%
\begin{tabular}{c|c|c|c|c|c|c|c|c|c|}
\cline{2-10}
                        & $V_L$ (V) & $V_F$ (V) & $I_L$ (A) & P (W) & Q (VAr) & S (VA) & FP    & $A_N$ (A)             & $V_{N'N}$  (V)        \\ \hline
\multicolumn{1}{|c|}{A} & 96,10     & 55,89     & 1,13      & 63,84 & 0,30    & 64,16  & 1     & \multirow{3}{*}{0,21} & \multirow{3}{*}{0} \\ \cline{1-8}
\multicolumn{1}{|c|}{B} & 100,07     & 56,57     & 0,62      & 22,12 & 27,68   & 35,58  & 0,625 &                       &                    \\ \cline{1-8}
\multicolumn{1}{|c|}{C} & 99,69     & 58,82     & 0,76      & 29,54 & 33,44   & 44,50  & 0,659 &                       &                    \\ \hline
\end{tabular}%
}
\end{table}

\begin{table}[H]
\centering \small \def\arraystretch{1.32}
\caption{Dados experimentais da primeira montagem em sequência CBA.}
\label{m1:dados:cba}
\resizebox{\textwidth}{!}{%
\begin{tabular}{c|c|c|c|c|c|c|c|c|c|}
\cline{2-10}
                        & $V_L$ (V) & $V_F$ (V) & $I_L$ (A) & P (W) & Q (VAr) & S (VA) & FP    & $A_N$ (A)            & $V_{N'N}$  (V)     \\ \hline
\multicolumn{1}{|c|}{A} & 100,50    & 57,60     & 1,096     & 63,25 & 0,24    & 63,51  & 1    & \multirow{3}{*}{1,6} & \multirow{3}{*}{0} \\ \cline{1-8}
\multicolumn{1}{|c|}{B} & 99,25     & 57,45     & 0,641     & 23,94 & 27,85   & 36,76  & 0,652 &                      &                    \\ \cline{1-8}
\multicolumn{1}{|c|}{C} & 100,70    & 58,34     & 0,767     & 30,57 & 32,61   & 55,52  & 0,683     &                      &                    \\ \hline
\end{tabular}%
}
\end{table}

\subsection{Carga em estrela com neutro isolado} \label{m2:dados}
Dos dados da Tabela \ref{m2:dados:abc}, ainda tem-se $P_T = 122,92$W, $Q_T = 55,16$VAr e $S_T = 149,04$VA. Enquanto que para sequência de fases CBA (Tabela \ref{m2:dados:cba}), $P_T = 125,598$W, $Q_T = 146,509$VAr e $S_T = 193,813$VA.

\begin{table}[H]
\centering \small \def\arraystretch{1.32}
\caption{Dados experimentais da segunda montagem em sequência ABC.}
\label{m2:dados:abc}
\begin{tabular}{c|c|c|c|c|c|c|c|c|}
\cline{2-9}
                        & $V_L$ (V) & $V_F$ (V) & $I_L$ (A) & P (W) & Q (VAr) & S (VA) & FP    & $V_{N'N}$ (V)      \\ \hline
\multicolumn{1}{|c|}{A} & 96,01     & 62,03     & 1,24      & 77,34 & 0,30    & 77,60  & 1     & \multirow{3}{*}{0} \\ \cline{1-8}
\multicolumn{1}{|c|}{B} & 100,6     & 54,74     & 0,59      & 20,24 & 25,60   & 32,78  & 0,621 &                    \\ \cline{1-8}
\multicolumn{1}{|c|}{C} & 99,51     & 54,98     & 0,70      & 25,34 & 29,26   & 38,66  & 0,654 &                    \\ \hline
\end{tabular}
\end{table}

\begin{table}[H]
\caption{Dados experimentais da segunda montagem em sequência CBA.}
\label{m2:dados:cba}
\centering \small \def\arraystretch{1.32}
\begin{tabular}{c|c|c|c|c|c|c|c|c|}
\cline{2-9}
                        & $V_L$ (V) & $V_F$ (V) & $I_L$ (A) & P (W) & Q (VAr) & S (VA) & FP    & $V_{N'N}$ (V)      \\ \hline
\multicolumn{1}{|c|}{A} & 100,4     & 14,18     & 0,210     & 3,968 & 0,069   & 3,963  & 1     & \multirow{3}{*}{42} \\ \cline{1-8}
\multicolumn{1}{|c|}{B} & 97,79     & 81,92     & 1,002     & 57,21 & 72,29   & 91,69  & 0,690 &                     \\ \cline{1-8}
\multicolumn{1}{|c|}{C} & 101,4     & 87,82     & 1,123     & 64,42 & 74,15   & 98,16  & 0,655 &                     \\ \hline
\end{tabular}
\end{table}

\subsection{Carga em triângulo desequilibrado} \label{m3:dados}
Dos dados da Tabela \ref{m3:dados:abc}, ainda tem-se $P_T = 88,603$W, $Q_T = 37,569$VAr e $S_T = 96,969$VA. Enquanto que para sequência de fases CBA (Tabela \ref{m3:dados:cba}), $P_T = 94$W, $Q_T = 4,303$VAr e $S_T = 91,19$VA.

\begin{table}[H]
\centering \small \def\arraystretch{1.32}
\caption{Dados experimentais da terceira montagem em sequência ABC.}
\label{m3:dados:abc}
\begin{tabular}{c|c|c|c|c|c|}
\cline{2-6}
                        & $I_L$ (A) & P (W) & Q (VAr) & S (VA) & FP    \\ \hline
\multicolumn{1}{|c|}{A} & 5,63      & 42,38 & 17,26   & 45,82  & 0,925 \\ \hline
\multicolumn{1}{|c|}{B} & 1,472     & 39,40 & 17,75   & 43,84  & 0,911 \\ \hline
\multicolumn{1}{|c|}{C} & 0,266     & 6,823 & 2,559   & 7,309  & 0,934 \\ \hline
\end{tabular}
\end{table}

\begin{table}[H]
\centering \small \def\arraystretch{1.32}
\caption{Dados experimentais da terceira montagem em sequência CBA.}
\label{m3:dados:cba}
\begin{tabular}{c|c|c|c|c|c|}
\cline{2-6}
 & $I_L$ (A) & P (W) & Q (VAr) & S (VA) & FP \\ \hline
\multicolumn{1}{|c|}{A} & 1,018 & 29,56 & 3,584 & 29,81 & 0,993 \\ \hline
\multicolumn{1}{|c|}{B} & 0,963 & 28,35 & 0,628 & 28,35 & 1 \\ \hline
\multicolumn{1}{|c|}{C} & 1,21 & 36,09 & 0,091 & 33,03 & 1 \\ \hline
\end{tabular}
\end{table}


\section{Análise sobre segurança} % 2,5%
Os óculos de segurança são Equipamentos de Proteção Individual (EPIs) e são utilizados para a proteção da área ao redor dos olhos contra qualquer tipo de detrito estranho, que possa causar irritação ou ferimentos. Também protegem contra faíscas, respingos de produtos químicos, detritos, poeira, radiação e etc \cite{safe}.
É importante a utilização desse equipamento durante os experimentos a fim de evitar qualquer dano, além de preparar o profissional para o manejo correto e seguro de qualquer equipamento.
Além disso, foi de extrema importância a presença do professor ou técnico na verificação da montagem do circuito antes de energizá-lo. Assim, reduziu-se riscos de curtos-circuitos ou sobrecarga na rede.


\section{Cálculos, análise dos resultados e questões} % (quando houver) (70%)

\subsection{Análise teórica do circuito}
Como o circuito é desequilibrado, a análise deve ser feita fase por fase. No entanto, há uma certeza: as tensões da fonte são equilibradas em módulo e ângulo. Assim, sabendo-se que $\mathrm{V_{L} =V_{F}\sqrt{3} \ \angle 30^{\circ }}$, tem-se os seguintes dados:

\vspace{-0.4cm}
\begin{equation*}
\begin{array}{ c c c }
\mathrm{\begin{bmatrix}
E_{AB}\\
E_{BC}\\
E_{CA}
\end{bmatrix} =100V\ \begin{bmatrix}
1\\
\alpha ^{2}\\
\alpha 
\end{bmatrix}} & \quad \text{e} \quad \  & \mathrm{\begin{bmatrix}
E_{AN}\\
E_{BN}\\
E_{CN}
\end{bmatrix} =57,74V\ \angle -30^{\circ } \begin{bmatrix}
1\\
\alpha ^{2}\\
\alpha 
\end{bmatrix}}
\end{array}
\end{equation*}

\vspace{0.2cm}
\subsubsection{Carga em estrela com neutro conectado} \label{m1:teoria}
Primeiramente, é possível descrever as impedâncias como abaixo:

\vspace{-0.5cm}
\begin{equation*}
\begin{aligned}
\begin{cases}
Z_{A} =50\ [ \Omega ]\\
Z_{B} =53,8+j\ 62,58\ [ \Omega ]\\
Z_{C} =50-j\ 57,79\ [ \Omega ]
\end{cases} & \quad \text{e também} \quad  & \begin{cases}
Y_{A} =0,02\ [ S]\\
Y_{B} =0,0122\ \angle -49,31\ [ S]\\
Y_{C} =0,0131\ \angle \ 49,13\ [ S]
\end{cases}
\end{aligned}
\end{equation*}
\vspace{0.2cm}

Como os neutros da fonte e da carga estão conectados, não há deslocamento de neutro, $V_{n'n}=0$. Portanto, a tensão na carga é a mesma da fonte, como consequência da presença do neutro.
\vspace{0.5cm}

\begin{equation*}
\begin{array}{l}
P_{A} =\Re e\left\{I_{A}^{*} \cdotp V_{AN}\right\} =I_{A} V_{AN} \cdotp cos\theta _{A}\\
Q_{A} =\operatorname{Im} \left\{I_{A} \ ^{*} \cdotp V_{AN}\right\} =I_{A} V_{AN} \cdotp sen\theta _{A}\\
S_{A} =I_{A} \ ^{*} \cdotp V_{AN}\\
\\
\text{Cálculo de corrente:}\ V=R\cdotp I\\
\left( 57,74\angle 0\mathrm{^{\circ }}\right) =Z_{A} \cdotp I_{A} \Rightarrow I_{A} =1,155\angle 0\mathrm{^{\circ }}[ A]\\
\left( 57,74\angle -120\mathrm{^{\circ }}\right) =Z_{B} \cdotp I_{B} \Rightarrow I_{B} =0,6997\angle -169,3\mathrm{^{\circ }} \ [ A]\\
\left( 57,74\angle 120\mathrm{^{\circ }}\right) =Z_{C} \cdotp I_{C} \Rightarrow I_{C} =0,756\angle 169,1\mathrm{^{\circ }}[ A]\\
\\
\begin{array}{l}
I_{N} =I_{B} +I_{C}\\
I_{N} =1,155\angle 0\mathrm{^{\circ }} +0,6997\angle -169,3\mathrm{^{\circ }} +0,756\angle 169,1\mathrm{^{\circ }}\\
I_{N} =0,275\angle 177,3\mathrm{^{\circ }}
\end{array}\\
\end{array}
\end{equation*}

\begin{equation*}
\begin{array}{l}
\mathrm{\begin{bmatrix}
I_{A}\\
I_{B}\\
I_{C}
\end{bmatrix}} =\mathrm{\begin{bmatrix}
1,155\angle 0\mathrm{^{\circ }}\\
0,6997\angle -169,3\mathrm{^{\circ }}\\
0,756\angle 169,1\mathrm{^{\circ }}
\end{bmatrix}}\\
\\
S_{A} =I_{A} \ ^{*} \cdotp V_{AN} =1,155\angle 0\mathrm{^{\circ }} \cdotp 57,74\angle 0\mathrm{^{\circ }} =66,69\angle 0\mathrm{^{\circ }}[ VA]\\
S_{B} =I_{B} \ ^{*} \cdotp V_{BN} =0,6997\angle -169,3^{\circ } \cdotp 57,74\angle 120^{\circ } =40,4\angle 70,7^{\circ }[ VA]\\
S_{C} =I_{C} \ ^{*} \cdotp V_{CN} =0,756\angle 169,1^{\circ } \cdotp 57,74\angle 120^{\circ } =43,65\angle -70,9^{\circ }[ VA]\\
\\
cos\theta _{A} =cos0^{\circ } =1\\
cos\theta _{B} =cos70,7^{\circ } =0,331\\
cos\theta _{C} =cos\left( -70,9^{\circ }\right) =0,327
\end{array}
\end{equation*}

\vspace{0.5cm}
A partir dos cálculos acima é possível montar uma tabela (Tabela \ref{m1:teorico:abc}) com os dados teóricos das grandezas de tensão e corrente no circuito. O cálculo é análogo para sequências de fases CBA, e os resultados são comtemplados pela Tabela \ref{m1:teorico:cba}.
\vspace{0.4cm}

\begin{table}[H]
\centering \small \def\arraystretch{1.32}
\caption{Dados teóricos para a montagem 1, em sequência ABC.}
\label{m1:teorico:abc}
\resizebox{\textwidth}{!}{%
\begin{tabular}{|c|c|c|c|c|c|c|c|c|c|}
\hline
 & $V_L$ (V) & $V_F$ (V) & $I_L$ (A) & P (W) & Q (VAr) & S (VA) & FP & $A_N$(A) & $V_{N'N}$ (V) \\ \hline
A & 100,0 & 57,74 & 1,155 & 66,69 & 0 & 66,69 & 1 & \multirow{3}{*}{0,275} & \multirow{3}{*}{0} \\ \cline{1-8}
B & 100,0 & 57,74 & 0,699 & 13,37 & 38,13 & 40,4 & 0,331 &  &  \\ \cline{1-8}
C & 100,0 & 57,74 & 0,756 & 14,27 & 41,25 & 43,65 & 0,327 &  &  \\ \hline
\end{tabular}%
}
\end{table}

\begin{table}[H]
\centering \small \def\arraystretch{1.32}
\caption{Dados teóricos para a montagem 1, em sequência CBA.}
\label{m1:teorico:cba}
\resizebox{\textwidth}{!}{%
\begin{tabular}{|c|c|c|c|c|c|c|c|c|c|}
\hline
 & $V_L$ (V) & $V_F$ (V) & $I_L$ (A) & P (W) & Q (VAr) & S (VA) & FP & $A_N$ (A) & $V_{N'N}$ (V) \\ \hline
A & 100,0 & 57,74 & 1,155 & 66,69 & 0 & 66,69 & 1 & \multirow{3}{*}{1,610} & \multirow{3}{*}{0} \\ \cline{1-8}
B & 100,0 & 57,74 & 0,663 & 24,96 & 29,03 & 38,28 & 0,652 &  &  \\ \cline{1-8}
C & 100,0 & 57,74 & 0,716 & 27,04 & 31,26 & 41,34 & 0,654 &  &  \\ \hline
\end{tabular}%
}
\end{table}
\vspace{0.3cm}

\subsubsection{Carga em estrela com neutro isolado} \label{m2:teoria}
Para esta montagem, como não há presença do fio neutro, haverá deslocamento de neutro, ou seja $V_{n'n}\ne 0$, como é mostrado nos cálculos abaixo. 

\begin{equation*}
\begin{array}{l}
V_{N'N} =\dfrac{E_{AN} \cdotp Y_{A} +E_{BN} +Y_{B} +E_{CN} \cdotp Y_{C}}{Y_{A} +Y_{B} +Y_{C}}\\
\\
V_{N'N} =\dfrac{57,74\angle 0\mathrm{^{\circ }} \cdotp 0,02+57,74\angle -120\mathrm{^{\circ }} \cdotp 0,0122\angle -49,31+57,74\angle 120\mathrm{^{\circ }} \cdotp 0,0131\angle 49,13\mathrm{^{\circ }}}{0,02+0,0122\angle -49,31\mathrm{^{\circ }} +0,0131\angle 49,13\mathrm{^{\circ }}}\\
\\
V_{N'N} =7,678\angle 176,5\mathrm{^{\circ }}
\end{array}
\end{equation*}

Assim, é possível realizar o cálculos das grandezas de tensão e corrente para sequências de fase ABC. A Tabela \ref{m2:teorico:abc} comtempla os resultados teóricos obtidos.
\vspace{0.3cm}

\begin{table}[H]
\centering \small \def\arraystretch{1.32}
\caption{Dados teóricos para a montagem 2, em sequência ABC.}
\label{m2:teorico:abc}
\resizebox{\textwidth}{!}{%
\begin{tabular}{|c|c|c|c|c|c|c|c|c|}
\hline
 & $V_L$ (V) & $V_F$ (V) & $I_L$ (A) & P (W) & Q (VAr) & S (VA) & FP & $V_{N'N}$ (V) \\ \hline
A & 100,0 & 65,41 & 1,308 & 85,56 & 0 & 85,56 & 1 & \multirow{3}{*}{7,678} \\ \cline{1-8}
B & 100,0 & 54,75 & 0,668 & 23,84 & 27,73 & 36,57 & 0,652 &  \\ \cline{1-8}
C & 100,0 & 53,88 & 0,706 & 24,88 & 28,77 & 38,04 & 0,654 &  \\ \hline
\end{tabular}%
}
\end{table}

\vspace{0.3cm}
Analogamente, faz-se os cálculo para sequência de fases CBA, e obtém-se os resultados da Tabela \ref{m2:teorico:cba}.

\begin{equation*}
\begin{array}{l}
V_{N'N} =\dfrac{E_{AN} \cdotp Y_{A} +E_{BN} +Y_{B} +E_{CN} \cdotp Y_{C}}{Y_{A} +Y_{B} +Y_{C}}\\
\\
V_{N'N} =\dfrac{57,74\angle 0\mathrm{^{\circ }} \cdotp 0,02+57,74\angle 120\mathrm{^{\circ }} \cdotp 0,0122\angle -49,31+57,74\angle -120\mathrm{^{\circ }} \cdotp 0,0131\angle 49,13\mathrm{^{\circ }}}{0,02+0,0122\angle -49,31\mathrm{^{\circ }} +0,0131\angle 49,13\mathrm{^{\circ }}}\\
\\
V_{N'N} =43,15\angle -2,839\mathrm{^{\circ }}
\end{array}
\end{equation*}

\begin{table}[H]
\centering \small \def\arraystretch{1.32}
\caption{Dados teóricos para a montagem 2, em sequência CBA.}
\label{m2:teorico:cba}
\resizebox{\textwidth}{!}{%
\begin{tabular}{|c|c|c|c|c|c|c|c|c|}
\hline
 & $V_L$ (V) & $V_F$ (V) & $I_L$ (A) & P (W) & Q (VAr) & S (VA) & FP & $V_{N'N}$ (V) \\ \hline
A & 100,0 & 11,84 & 0,237 & 2,086 & 0 & 2,806 & 1 & \multirow{3}{*}{43,15} \\ \cline{1-8}
B & 100,0 & 86,14 & 1,051 & 44,54 & 78,79 & 90,53 & 0,492 &  \\ \cline{1-8}
C & 100,0 & 83,76 & 1,097 & 38,13 & 83,61 & 91,88 & 0,415 &  \\ \hline
\end{tabular}%
}
\end{table}
\vspace{0.2cm}

\subsubsection{Carga em triângulo desequilibrado} \label{m3:teoria}
Para carga em triângulo, tem-se as impedâncias e admitâncias descritas abaixo.

\vspace{-0.5cm}
\begin{equation*}
\begin{aligned}
\begin{cases}
Z_{AB} =50\ [ \Omega ]\\
Z_{BC} =53,8+j\ 62,58\ [ \Omega ]\\
Z_{CA} =50-j\ 57,79\ [ \Omega ]
\end{cases} & \quad \text{e também} \quad  & \begin{cases}
Y_{AB} =0,02\ [ S]\\
Y_{BC} =0,0122\ \angle -49,31\ [ S]\\
Y_{CA} =0,0131\ \angle \ 49,13\ [ S]
\end{cases}
\end{aligned}
\end{equation*}
\vspace{0.2cm}

Ademais, com o intuito de manter as correntes de linha com módulos inferiores a 2A, a tensão de linha passa a ser de 50V. Logo tem-se:


\vspace{-0.4cm}
\begin{equation*}
\begin{array}{ c c c }
\mathrm{\begin{bmatrix}
E_{AB}\\
E_{BC}\\
E_{CA}
\end{bmatrix} =50V\ \begin{bmatrix}
1\\
\alpha ^{2}\\
\alpha 
\end{bmatrix}} & \quad \text{e} \quad \  & \mathrm{\begin{bmatrix}
E_{AN}\\
E_{BN}\\
E_{CN}
\end{bmatrix} =28,87V\ \angle -30^{\circ } \begin{bmatrix}
1\\
\alpha ^{2}\\
\alpha 
\end{bmatrix}}
\end{array}
\end{equation*}
\vspace{0.2cm}

Assim, é possível calcular e organizar os dados teóricos para as sequências ABC e CBA, respectivamente, nas Tabelas \ref{m3:teorico:abc} e \ref{m3:teorico:cba}.
\vspace{0.2cm}

\begin{equation*}
\begin{array}{l}
I_{AB} =\dfrac{V_{AB}}{Z_{AB}} =\dfrac{50\angle 0\mathrm{^{\circ }}}{50} =1\angle 0\mathrm{^{\circ }}\\
\\
I_{BC} =\dfrac{V_{BC}}{Z_{BC}} =\dfrac{50\angle 120\mathrm{^{\circ }}}{53,8+j\ 62,58} =0,929\angle 57,4\mathrm{^{\circ }}\\
\\
I_{AC} =\dfrac{V_{CA}}{Z_{CA}} =\dfrac{50\angle -120\mathrm{^{\circ }}}{50-j\ 57,79} =0,654\angle -70,87\mathrm{^{\circ }}\\
\\
I_{A} =I_{AB} -I_{CA} =1\angle 38,18\mathrm{^{\circ }}\\
I_{B} =\ I_{BC} -I_{AB} =0,928\angle 122,5\mathrm{^{\circ }}\\
I_{C} =I_{CA} -I_{BC} =1,429\angle -101,5\mathrm{^{\circ }}
\end{array}
\end{equation*}
\vspace{0.3cm}

%% CORRENTES AB  
%% TABELAS ABC E CBA
\begin{table}[H]
\centering \small \def\arraystretch{1.32}
\caption{Dados teóricos para a montagem 3, em sequência ABC.}
\label{m3:teorico:abc}
\begin{tabular}{|c|c|c|c|c|c|}
\hline
 & $I_L$ (A) & P (W) & Q (VAr) & S (VA) & FP \\ \hline
A & 1,647 & 39,28 & -26,80 & 47,55 & 0,826 \\ \hline
B & 1,599 & 41,50 & 20,24 & 46,16 & 0,899 \\ \hline
C & 0,241 & 6,828 & -1,352 & 6,96 & 0,981 \\ \hline
\end{tabular}
\end{table}

\begin{table}[H]
\centering \small \def\arraystretch{1.32}
\caption{Dados teóricos para a montagem 3, em sequência CBA.}
\label{m3:teorico:cba}
\begin{tabular}{|c|c|c|c|c|c|}
\hline
 & $I_L$ (A) & P (W) & Q (VAr) & S (VA) & FP \\ \hline
A & 1 & 10,74 & -26,80 & 28,87 & 0,372 \\ \hline
B & 0,928 & 22,58 & -14,39 & 26,79 & 0,843 \\ \hline
C & 1,429 & 27,36 & -30,9 & 41,26 & 0,663 \\ \hline
\end{tabular}
\end{table}

\vspace{0.2cm}

\subsection{Reflexão}
\subsubsection{E na ausência de um voltímetro? ...}
% - Na impossibilidade de ter voltímetros, amperímetros e sequencímetro, como você poderia encontrar a solução desse problema (abc ou cba)?
 

\subsubsection{Mas porque encontrar a correta sequência de fase em um circuito elétrico? }
% - Aponte a importância de encontrar a correta sequência de fase em um circuito elétrico.



\newpage
\section{Simulação computacional} % (10%);
Para sequência de fases ABC tem-se: 

\begin{figure}[H]
\centering
\includegraphics[width=13.5cm]{m3-esquema-abc-correntes}
\caption{Correntes de linha da montagem 3 em sequência de fases ABC.}
\label{m2:IL}
\end{figure}

\vspace{0.3cm}
Para sequência de fases CBA tem-se: 

\begin{figure}[H]
\centering
\includegraphics[width=13.5cm]{m3-esquema-cba-correntes}
\caption{Correntes de linha da montagem 3 em sequência de fases CBA.}
\label{m2:IL}
\end{figure}


\newpage
\section{Conclusões} % (no mínimo 10 linhas) (5%);

\newpage
\begin{thebibliography}{9} 
% Introdução
\bibitem{PH}
    P. H. O. Rezende,
    "Circuitos Polifásicos Desequilibrados", 2018.

\bibitem{safe}
    SafetyTrabi,
    “Óculos de segurança: Saiba quando utilizar este EPI”, SafetyTrab, 2019.
 Disponível em:
 \url{https://www.safetytrab.com.br/blog/oculos-de-seguranca/}. Acesso em: ago. 2019.

\bibitem{UNIR}
    B. M. Nascimento, J. Carneiro, P. Viecilli,
    “Sequencímetro de Baixo Custo”, Fundação Universidade Federal de Rondônia - UNIR.
 Disponível em:
 \url{https://brunomarquesunir.wixsite.com/sequencimetro}. Acesso em: dez. 2019.

\bibitem{fig1}
    Dutra Máquinas.
 Disponível em:
 \url{https://m.dutramaquinas.com.br/p/fasimetro-com-indicador-led-690-volts-mfa-862-mfa-862}. Acesso em: dez. 2019.


\end{thebibliography}
\end{document}
