\documentclass[a4paper,12pt,oneside,openany,table,xcdraw]{article}

\usepackage{setspace}
\usepackage{multirow}
\usepackage{hyperref}
\usepackage{caption}
\usepackage{indentfirst}

\usepackage[brazilian]{babel}
\usepackage[utf8x]{inputenc}
\usepackage{amsmath, graphicx, enumerate}
\usepackage{float, verbatim}
\usepackage[colorinlistoftodos]{todonotes}
\usepackage{makeidx} % Para o sumário
\usepackage{geometry}

\geometry{a4paper, hmargin={3cm, 3cm}, vmargin={3cm, 2cm} }
\setlength{\parindent}{1.0cm}

\begin{document}
\newcommand{\thedepartment}{Faculdade de Engenharia Elétrica}
\newcommand{\thecourse}{FEELT}
\newcommand{\thetitle}{FONTE}
\newcommand{\thetype}{Relatório da Disciplina de Circuitos Elétricos II}
\newcommand{\theproftitle}{Bacharel em Engenharia Elétrica}
\newcommand{\thestudent}{Lesly Viviane Montúfar Berrios\\
\centering11811ETE001}
\newcommand{\theadvisor}{Prof. Wellington Maycon Santos Bernardes}
\newcommand{\thecity}{Uberlândia}

\thispagestyle{empty}\newcommand*{\themonth}{\ifthenelse{\the\month < 2}{Janeiro }
                  {\ifthenelse{\the\month < 3}{Fevereiro }
                  {\ifthenelse{\the\month < 4}{Março }
                  {\ifthenelse{\the\month < 5}{Abril }
                  {\ifthenelse{\the\month < 6}{Maio }
                  {\ifthenelse{\the\month < 7}{Junho }
                  {\ifthenelse{\the\month < 8}{Julho }
                  {\ifthenelse{\the\month < 9}{Agosto }
                  {\ifthenelse{\the\month < 10}{Setembro }
                  {\ifthenelse{\the\month < 11}{Outubro }
                  {\ifthenelse{\the\month < 12}{Novembro }{Dezembro }}}}}}}}}}}}
                  
\begin{titlepage}
\begin{center}

	\vspace{-0.5cm}

  \begin{figure}[hbt!]
		\begin{center}
		   \includegraphics[width=2.8cm]{ufu-logo.png}
		\end{center}
	\end{figure}
 	%\vspace{-4cm}

%\begin{doublespacing}

  \Large{\textbf{Universidade Federal de Uberlândia}}\\
  \large{\thedepartment}\\
  \large{\thecourse}\\


\vspace{6cm}
  \par
  \large\textbf{\thetitle}
\vspace{6cm} 

%\end{doublespacing}
  \par
  \thetype\\
  por\\
  %\hspace{2cm}\large{}\\

\vspace{1cm}
\par\vfil
  \normalsize{\thestudent}\\ [2cm]
  \theadvisor

\par\vfill
  \thecity, \themonth / \the\year

\end{center}

\end{titlepage}

%% Comeca o documento !

\onehalfspacing
\tableofcontents % sumário
\newpage

\section{Introodução}
% A importancia da fonte

\section{Componentes}

\section{Funcionamento do circuito}
\subsection{Retificação}
\subsection{Filtragem}
\subsection{Regulação}


\section{Memória de Cálculo}

\newpage
\begin{thebibliography}{9} 
% Introdução
\bibitem{irwin}
    J. D. Irwin,
    “Análise de Circuitos Em Engenharia”, Pearson, $4^a$ Ed., 2000.

\bibitem{boylestad}
    R. L. Boylestad,
    “Introdução À Análise de Circuitos”, Pearson, $10^a$ Ed., 2004.

\bibitem{safe}
    SafetyTrabi,
    “Óculos de segurança: Saiba quando utilizar este EPI”, SafetyTrab, 2019.
 Disponível em:
 \url{https://www.safetytrab.com.br/blog/oculos-de-seguranca/}. Acesso em: ago. 2019.


\end{thebibliography}
\end{document}
